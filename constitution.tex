% arara: pdflatex
% arara: pdflatex
% arara: pdflatex: { synctex: true }

\documentclass[letterpaper,12pt]{scrartcl}

\usepackage[T1]{fontenc}

\usepackage{libertine}

\usepackage{microtype}

\usepackage[american]{babel}
\usepackage{csquotes}

\usepackage[margin=1in]{geometry}

\usepackage{authoraftertitle}
\author{Oklahoma State University Amateur Radio Club}
\title{OSU ARC Constitution}

\usepackage[hidelinks]{hyperref}

\hypersetup{
  pdftitle    = {\MyTitle},
  pdfauthor   = {\MyAuthor},
  pdfsubject  = {OSU Student Organization}
}

\newcommand{\myarticle}[2][art:\arabic{section}]{%
  \phantomsection
  \refstepcounter{section}%
  \label{art:#1}%
  \addcontentsline{toc}{section}{%
    \sectionautorefname{} \Roman{section}. #2%
  }%
  \section*{%
    \sectionautorefname{} \Roman{section}. #2
  }%
}

\newcommand{\mysection}[2][art:\arabic{section}:\Alph{subsection}]{%
  \phantomsection
  \refstepcounter{subsection}%
  \label{sec:#1}%
  \addcontentsline{toc}{subsection}{%
    \textit{\subsectionautorefname{} \Alph{subsection}: #2}%
  }%
  \subsection*{%
    \textit{\subsectionautorefname{} \Alph{subsection}: #2}%
  }%
}

\newcommand{\mysubsection}[2][art:\arabic{section}:\Alph{subsection}:\arabic{subsubsection}]{%
  \phantomsection
  \refstepcounter{subsubsection}%
  \label{sub:#1}%
  \addcontentsline{toc}{subsubsection}{%
    \subsubsectionautorefname{} \arabic{subsubsection}: #2%
  }%
  \subsubsection*{%
    \subsubsectionautorefname{} \arabic{subsubsection}: #2%
  }%
}

\renewcommand\thesection{\Roman{section}}
\renewcommand\thesubsection{\Alph{subsection}}
\renewcommand\thesubsubsection{\arabic{subsubsection}}

\begin{document}

\def\sectionautorefname{Article}
\def\subsectionautorefname{Section}
\def\subsubsectionautorefname{Subsection}

\tableofcontents

\cleardoublepage

\phantomsection
\addcontentsline{toc}{section}{Preamble}
\section*{Preamble\label{preamble}}
We the students of Oklahoma State University constitute ourselves as the Oklahoma State University Amateur Radio Club. The purpose of the Oklahoma State University Amateur Radio Club shall be to promote interest in Amateur Radio and the development of the practice of Amateur Radio.

\myarticle[one]{}
Possession of a valid amateur radio license or lack thereof shall not affect the rank of any member with the exception of officers. Officers must possess a valid amateur radio license as outlined in \autoref{art:officers}.

\mysection[fullmems]{Full Members}
Membership in any student organization is limited to students only, and those students must meet the minimum standards as required in the Student Rights and Responsibilities Governing Student Behavior (SRR XI.F.6). An individual is a member if they have paid all applicable dues to the Oklahoma State University Amateur Radio Club (OSU ARC).

\mysubsection[memvote]{Voting}
Full members shall have full voting privileges contingent upon payment of dues for the current dues term.

\mysubsection[memprivs]{Privileges}
Full members shall be allowed to use the OSU ARC radio equipment according to the stated operator privileges of their Federal Communications Commission (FCC) amateur radio license or the operator privileges granted to their non-US amateur radio license by the appropriate reciprocal operating agreements and/or operating permits.

\mysection[assmems]{Associate Members}
Faculty, academic staff, and non-academic staff of Oklahoma State University may be associate members in the OSU ARC.

\mysubsection[assvote]{Voting}
Associate members have no voting privileges on any Club business per Student Government Association guidelines.

\mysubsection[assprivs]{Privileges}
Associate members are required to pay dues to the OSU ARC as set forth in the Club by-laws.

\mysection[honmems]{Honorary Members}
Persons not affiliated with the University may become honorary members of the OSU ARC if authorized by the Office of the Vice President for Student Affairs (SRR XI.F.2).

\mysubsection[honvote]{Voting}
Honorary members have no voting privileges on any Club business per Student Government Association guidelines.

\mysubsection[honprivs]{Privileges}
Honorary members are allowed to attend Club meetings and functions. Honorary members are not charged Club dues.

\mysection[privloss]{Loss of Privileges}
\begin{enumerate}
\item Any member who has not paid their applicable dues by the meeting following their first meeting as a member shall have their membership privileges immediately revoked.
\item Any member unable to pay their Club dues should petition the Executive Committee for a waiver as soon as possible. The petition shall consist of a written statement explaining the reason for the desired waiver.
\item If a member fails to meet the requirements of membership their privileges shall be revoked, and they shall be required to apply for a different level of membership.
\item If any member breaks the Ham Radio Office Rules of Conduct, as stated in the by-laws, appropriate punishment shall be determined by the Executive Committee.
\end{enumerate}

\myarticle[officers]{Officers}

\mysection[execcommittee]{Executive Committee}
The Executive Committee of the OSU ARC shall consist of the President, Vice President, Treasurer, Secretary, and the College of Engineering, Architecture, and Technology (CEAT) Student Council Representative.

\mysection[officeequal]{Qualification for Officers}
To be eligible for office within a student organization an undergraduate student must maintain a minimum 2.0 grade point average and be enrolled full-time (12 hours or more). A graduate/professional student must maintain a minimum 3.0 grade point average and be enrolled full-time (SRR XI.F.7). In addition the OSU ARC requires possession of a valid FCC amateur radio license or a valid non-US amateur radio license with permission to operate within the United States.

\mysection[electionoffice]{Election of Officers}
Officer election cannot occur any earlier than the third to the last Club meeting of the academic year and must be announced two regular club meetings in advance. Nominations may start at the same time and all nominees must meet the qualifications set forth in \hyperref[sec:officeequal]{\autoref*{art:officers}, \autoref*{sec:officeequal}}. Newly elected officers take their positions immediately. Officers are elected to a one year term with no term limits.

\mysection[officeremoval]{Removal from Office}
In the event that an officer ceases to be a full fember of the OSU ARC, ceases to be a full-time student at the University, fails to maintain a satisfactory grade point average, or is put on any type of probation with the University, that person's office shall become vacant and a replacement elected according to \hyperref[sec:vacancies]{\autoref*{art:officers}, \autoref*{sec:vacancies}} of this Constitution.

\mysection[impeachment]{Impeachment}
Impeachment charges may be brought against any officer by the submition of a Petition for Impeachment to the Executive Committee, signed by at least one-half of all full members. The officer against whom charges are brought may be impeached at the next regularly scheduled meeting by a three-quarters majority vote of all full members present. An officer who is so impeached shall be removed from office immediately and a replacement elected according to the terms of \hyperref[sec:electionoffice]{\autoref*{art:officers}, \autoref*{sec:electionoffice}} of this Constitution. An impeachment vote shall not affect the membership status of the member concerned.

\mysection[vacancies]{Vacancies}
In the event of a vacancy in the office of President the Vice President shall assume the office of President, and a new Vice President shall be elected at the following meeting. Vacancies in any other office shall be filled by a special election at the first meeting following the vacating of the office. Until such election occurs the President may appoint a member as an acting officer to fill the vacancy.

\mysection[officerduties]{Duties of Officers}
All officers are responsible for submitting a document outlining their goals for the Club for the current academic year. At the end of the year all officers are responsible for submitting a written report of accomplishments made and problems encountered during the preceding term and make any suggestions for the future.

\mysubsection[president]{President}
The President shall preside over all meetings and shall represent the Club in all official matters.

\mysubsection[vicepresident]{Vice President}
The Vice President shall assume the duties of the President in the event of the President's absence. The Vice President shall also preside over all Club committees.

\mysubsection[treasurer]{Treasurer}
The Treasurer shall maintain an accurate record of the Club's finances. He or she will also be responsible for completion of all documentation necessary to reimburse or pay a party.

\mysubsection[secretary]{Secretary}
The Secretary shall record and maintain accurate minutes of every meeting and make them available to the Club within one week of the meeting. The Secretary shall also read the minutes of the previous meeting at the following Club meeting.

\mysubsection[councilrep]{CEAT Student Council Representative}
The College of Engineering, Architecture, and Technology (CEAT) Student Council Representative shall be the liaison between the Council and the OSU ARC. The CEAT Student Council Representative shall also provide a report at every meeting of any information regarding CEAT.

\myarticle[meetings]{Meetings}

\mysection[location]{Location, Time, and Procedure}
The location and the time of the Club's meetings shall be set forth in the by-laws of the Club. Unless otherwise stated in the by-laws the Club's meetings shall follow standard parliamentary procedure according to the latest edition of Robert's Rules of Order. The Club's Secretary shall act as the parliamentarian during all Club meetings. Parliamentary disputes shall be resolved by the Secretary, who may consult the President or Faculty Advisor if a dispute arises.

\mysection[conductbusiness]{Conducting Business}
A minimum of 1/3 of the full members shall constitute a quorum for the transaction of business. No official Club business may be conducted when a quorum is not present.

\mysection[voting]{Voting Procedures}
Unless otherwise stated in this Constitution, the outcome of any vote shall be determined by a majority (51\%) of voting members. An absence shall be counted as an abstention from voting. In the election of officers, if no candidate receives more than half of the votes cast, all nominations for that office shall be void and nominations for that office shall be reopened. Ballots for each office shall be taken independently of other offices.

\mysection[onlinevoting]{Online Voting Procedures}
At the option of the Club officers, an online voting system may be used instead of a standard system. The online system must abide by the voting procedures in \hyperref[sec:voting]{\autoref*{art:meetings}, \autoref*{sec:voting}} and have the ability to ensure that eligible Club members are only able to vote once.

\myarticle[bylaws]{By-Laws}

\mysection[clubauthority]{Authority}
The Club shall have the authority to pass by-laws that will carry out the terms of this Constitution and the normal operation of the Club. No by-law shall be passed which violates this Constitution, the regulations of the University, or any local, state, or federal law.

\mysection[minlaw]{Minimum Contents of the By-Laws}
The by-laws, at minimum, shall contain the following: Club dues, meeting dates and times, procedures for obtaining access to Club equipment, and procedures for the use and modification of Club equipment.

\mysection[passinglaw]{Passing a By-Law}
A by-law may be passed at any regular Club meeting by a simple majority vote when a quorum of voting members is present.

\mysection[lawvalidity]{Validity of By-Laws}
On the motion of five members that a by-law is unconstitutional a Judicial Committee composed of the members of the Executive Committee - with the exception of the President - shall review the by-law and determine its constitutionality by a simple majority vote. Deliberations of this Judicial Committee must be made in open session. Club members must be informed of the time and place of the Judicial Committee's hearings. If a majority of the membership is not satisfied with any decision of the Judicial Committee an appeal may be made as outlined in \autoref{art:subordination} of this constitution.

\myarticle[equipment]{Equipment}
All equipment shall be the property of the Department of Electrical and Computer Engineering, College of Engineering, Architecture, and Technology, Oklahoma State University. No equipment shall be removed from the Club premises or the equipment's designated operating location without the approval of the Faculty Advisor or two club officers. Any matters of access to the Ham Radio Office shall be decided by the head of the Department of Electrical and Computer Engineering.

\myarticle[enforcement]{Enforcement}

\mysection[violations]{Violations}
If at any time the Faculty Advisor or a majority of the Executive Committee believes that a club member has violated, attempted to violate, or intends to violate any local, state, or federal law, University regulation, provision of this Constitution, or by-law in connection with the OSU ARC, that member may be put on immediate probation. The person or persons placing the member on probation may at any time terminate the probationary status of the member if the probation order is found to be invoked in error.

\mysection[probation]{Probation}
A member on probation shall have all privileges granted to them in \autoref{art:one} of this constitution suspended with the exception of voting rights. Members on probation may have their voting rights suspended at the discretion of the Club officers. The length of the probationary period shall be determined by the Executive Committee and the Faculty Advisor.

\mysection[punitives]{Punitive measures}
If a member of the Club violates any local, state, or federal law they will be dealt with according to University and law enforcement authorities. Otherwise they will be dealt with on a case by case basis by the Executive Committee and the Faculty Advisor. Penalties may include, but are not limited to, loss of all member privileges, loss of voting rights, access to Club offices and equipment, and expulsion from the Club.

\mysection[facultyauthority]{Authority of faculty advisor}
The Faculty Advisor has the right to remove anyone from the Club at any time at his or her discretion.

\myarticle[reports]{Reports to the University}

\mysection[reportofficers]{Officers}
Within two weeks of the new officers entering office, and at the beginning of each academic year, the OSU ARC shall report a list of its officers and members to the Student Activities Office. The President, Treasurer, and Faculty Advisor shall file a Signature Authorization Card in the same office.

\mysection[financialstatements]{Financial statements}
At the request of the Student Government Association, the OSU ARC shall submit a report of its finances and/or financial transactions.

\mysection[failuretocomply]{Failure to comply}
Failure to render such reports, either through continued negligence or willful omission, may result in the suspension of the Club's charter.

\myarticle[subordination]{Subordination to the SGA Constitution}
This constitution shall not supersede any part or parts of the Constitution of the Student Government Association of Oklahoma State University. This organization shall be under the jurisdiction of the Student Government Association.

\myarticle[interpconst]{Interpretation of this Constitution}
In the event that a disagreement arises regarding the interpretation of this Constitution the Executive Committee and Faculty Advisor shall attempt an initial resolution. If the Executive Committee and Faculty Advisor fail to reach a conclusion the conflict shall be submitted to the Student Government Association and resolved according to their constitution.

\myarticle[amendments]{Amendments}
Any new constitution, or any proposed amendment to this Constitution, must be proposed in the exact written form to be considered in a regularly scheduled club meeting at least two weeks, but not more than six weeks, before it is to be voted upon. In voting on any new constitution or amendment to this Constitution, a roll-call ballot of voting members shall be taken in open session. An acceptance vote of three-quarters of the total number of voting members must be given for passage. The new constitution or amendment shall not be valid until it is ratified by the Student Government Association.

\end{document}
