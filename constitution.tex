% arara: pdflatex
% arara: pdflatex
% arara: pdflatex: { synctex: true }

\documentclass[letterpaper,12pt]{scrartcl}

\usepackage[nofontspec]{libertine}

\usepackage{microtype}

\usepackage[american]{babel}
\usepackage{csquotes}

\usepackage[margin=1in]{geometry}

\usepackage{authoraftertitle}
\author{Oklahoma State University Amateur Radio Club}
\title{OSU ARC Constitution}

\usepackage[hidelinks]{hyperref}

\hypersetup{
  pdftitle    = {\MyTitle},
  pdfauthor   = {\MyAuthor},
  pdfsubject  = {OSU Student Organization}
}

\newcommand{\myarticle}[2][art:\arabic{section}]{%
\phantomsection
\refstepcounter{section}%
\label{art:#1}%
\addcontentsline{toc}{section}{Article \Roman{section}. #2}%
\section*{Article \Roman{section}. #2}%
}

\newcommand{\mysection}[2][art:\arabic{section}:\Alph{subsection}]{%
\phantomsection
\refstepcounter{subsection}%
\label{sec:#1}%
\addcontentsline{toc}{subsection}{\textit{Section \Alph{subsection}: #2}}%
\subsection*{\textit{Section \Alph{subsection}: #2}}%
}

\newcommand{\mysubsection}[2][art:\arabic{section}:\Alph{subsection}:\arabic{subsubsection}]{%
\phantomsection
\refstepcounter{subsubsection}%
\label{sub:#1}%
\addcontentsline{toc}{subsubsection}{Subsection \arabic{subsubsection}: #2}%
\subsubsection*{Subsection \arabic{subsubsection}: #2}%
}

\renewcommand\thesection{\Roman{section}}
\renewcommand\thesubsection{\Alph{subsection}}
\renewcommand\thesubsubsection{\arabic{subsubsection}}

\begin{document}

\def\sectionautorefname{Article}
\def\subsectionautorefname{Section}
\def\subsubsectionautorefname{Subsection}

\tableofcontents

\cleardoublepage

\phantomsection
\addcontentsline{toc}{section}{Preamble}
\section*{Preamble\label{preamble}}
We, the students of Oklahoma State University, constitute ourselves as the Oklahoma State University Amateur Radio Club, hereby known as the OSU ARC. The purposes of the OSU ARC shall be to promote interest in Amateur Radio and the development of the practice of Amateur Radio.

\myarticle[one]{}
Possession of an amateur radio license shall not affect the rank of any member, except for running for club office where a license is necessary as outlined in \autoref{art:officers}.

\mysection[fullmems]{Full Members}
Membership in any student organization is limited to students only and those students must meet the minimum standards as required in the Student Rights and Responsibilities Governing Student Behavior (SRR XI.F.6). An Individual is a full member if they have paid their club dues.

\mysubsection[memvote]{Voting}
Full members shall have full voting privileges pending they have paid their dues for the current due term.

\mysubsection[memprivs]{Privileges}
Full members shall be allowed to use the radio equipment according to the regulations on their Amateur Radio license set forth by the Federal Communications Commission (FCC).

\mysection[assmems]{Associate Members}
University faculty or non-faculty staff employees may be associate members in the OSU ARC.

\mysubsection[assvote]{Voting}
Associate members shall have full voting privileges pending they have paid their dues for the current due term.

\mysubsection[assprivs]{Privileges}
Associate members are permitted to pay the club dues set forth by the club by-laws.

\mysection[honmems]{Honorary Members}
Persons not affiliated with the University may also become honorary members of the OSU ARC if authorized by the Office of the Vice President for Student Affairs (SRR XI.F.2).

\mysubsection[honvote]{Voting}
Honorary members have no voting privileges on any club business according to OSU SGA guidelines.

\mysubsection[honprivs]{Privileges}
Honorary members are allowed to attend club meetings and functions. Honorary members are not allowed to pay club dues.

\mysection[privloss]{Loss of Privileges}
\begin{enumerate}
\item Any member who has not paid their dues at the meeting following their first meeting attended, he or she will lose membership privileges immediately.
\item Any member unable to pay their club dues should petition the Executive Committee as soon as possible for a waiver. The petition shall be a written statement explaining the reason for the desired waiver.
\item If a member fails to meet the requirements of membership their privileges shall be revoked and they will have to apply for a different level of membership.
\item If any member breaks the Ham Radio Office Rules of Conduct, as stated in the by-laws, appropriate punishment will be determined by the executive committee.
\end{enumerate}

\myarticle[officers]{Officers}

\mysection[execcommittee]{Executive Committee}
The Executive Committee of the OSU ARC shall consist of the President; Vice President; Treasurer; Secretary; and the College of Engineering, Architecture, and Technology (CEAT) Student Council Representative.

\mysection[officeequal]{Qualification for Officers}
To be eligible for office within a student organization, an undergraduate student must maintain a 2.0 grade point average and be enrolled in a full course of study (12 hours). A graduate/professional student must be in good academic standing and be enrolled in a full course of study (SRR XI.F.7). Possession of an amateur radio license is necessary to qualify for an office.

\mysection[electionoffice]{Election of Officers}
Officer election cannot occur any earlier than the third to the last club meeting of the year and must be announced two regular club meetings in advance. Nominations may start at the same time and all nominees must meet the qualifications set forth in \hyperref[sec:officeequal]{\autoref*{art:officers}, \autoref*{sec:officeequal}}. Officers elected take their positions immediately. Officers are elected for one year and can serve any number of consecutive terms.

\mysection[officeremoval]{Removal from Office}
In the even that an officer ceases to be a Full Member of the OSU ARC, ceases to be a full-time student in the University, no longer has a satisfactory GPA, or is put on any type of probation with the University, that person's office shall become vacant and a replacement shall be elected according to \hyperref[sec:vacancies]{\autoref*{art:officers}, \autoref*{sec:vacancies}} of this constitution.

\mysection[impeachment]{Impeachment}
Impeachment charges may be brought against any officer by submitting a petition, signed by at least one-half of the full members, to the Executive Committee. The officer against whom charges are brought may be impeached at the next regularly scheduled meetings by a three-quarters majority vote of all members present. An officer who is so impeached shall be immediately removed from office and a replacement shall be elected according to the terms of \hyperref[sec:electionoffice]{\autoref*{art:officers}, \autoref*{sec:electionoffice}} of this constitution. An impeachment vote shall not affect the membership status of the member.

\mysection[vacancies]{Vacancies}
All officers are responsible for submitting a document outlining their goals for the club for this year. At the end of the year, all officers are responsible for submitting a written report of the problems and accomplishments made during the preceding term and make any suggestions for the future.

\mysection[officerduties]{Duties of Officers}
All officers are responsible for submitting a document outlining their goals for the club for this year. At the end of the year, all officers are responsible for submitting a written report of the problems and accomplishments made during the preceding term and make any suggestions for the future.

\mysubsection[president]{President}
The President shall preside over all meetings and shall represent the club in all official matters.

\mysubsection[vicepresident]{Vice President}
The Vice President shall assume the duties of the President in the event of the President's absence. The Vice President shall also preside over all club committees.

\mysubsection[treasurer]{Treasurer}
The Treasurer shall maintain an accurate record of the club's finances. He or she will also be responsible for completion of all documentation necessary to reimburse or pay a party.

\mysubsection[secretary]{Secretary}
The Secretary shall record and maintain accurate minutes of every meeting and make them available to the club within one week of the meeting. The Secretary shall also read the minutes of the previous meeting at the next club meeting.

\mysubsection[councilrep]{CEAT Student Council Representative}
College of Engineering, Architecture, and Technology Student Council Representative The College of Engineering, Architecture, and Technology Student Council Representative shall be the liaison between the Council and the OSU ARC. The Student Council Representative shall also provide a report at every meeting of any information regarding CEAT.

\myarticle[meetings]{Meetings}

\mysection[location]{Location, Time, and Procedure}
The location and the time of the club's meetings will be set forth by the bylaws of the organization. Unless otherwise stated in the bylaws, the club's meetings shall follow standard parliamentary procedure according to the latest edition of Robert's Rules of Order. The club's Secretary shall act as the parliamentarian during all club meetings. Parliamentary disputes shall be resolved by the Secretary, who may consult the President or faculty advisor if a dispute arises.

\mysection[conductbusiness]{Conducting Business}
A minimum of 1/3 of the full members shall constitute a quorum for the transaction of business. No official club business may be conducted when a quorum is not present.

\mysection[voting]{Voting Procedures}
Unless otherwise stated in this constitution, the outcome of any vote will be determined by a majority (51\%) of voting members. An absence shall be counted as not casting a vote. In the election of officers, if no candidate receives more than half of the votes cast, all nominations for that office shall be void and nominations for that office shall be reopened. Ballots for each office shall be taken independently of other offices.

\mysection[onlinevoting]{Online Voting Procedures}
At the option of the club officers, an online voting system may be used instead of a standard system. The online system must abide by the voting procedures in \hyperref[sec:voting]{\autoref*{art:meetings}, \autoref*{sec:voting}} and have the ability to ensure that club members are only able to vote once.

\myarticle[bylaws]{By-Laws}

\mysection[clubauthority]{Authority}
The club shall have the authority to pass by-laws that will carry out the terms of this constitution and the normal operation of the club. No by-law shall be passed which violates this constitution, the regulations of Oklahoma State University, or any law.

\mysection[minlaw]{Minimum Contents of the By-Laws}
The by-laws, at minimum, shall contain the following: club dues, meeting dates and times, procedures for obtaining access to club equipment, and procedures for the use and modifications of club equipment.

\mysection[passinglaw]{Passing a By-Law}
A by-law may be passed at any regular club meeting by a majority vote when a quorum of voting members is present.

\mysection[lawvalidity]{Validity of By-Laws}
On the motion of five members that a by-law is unconstitutional, a judicial committee composed of the members of the executive committee except the President shall review the by-law and determine its constitutionality by a simple majority vote. Deliberations of this judicial committee must by made in open session. The members off the club must be informed of the time and place of the judicial committee's hearings. If a majority of the membership is not satisfied with any decision of the Judicial Committee an appeal may be made as outlined in \autoref{art:subordination} of this constitution.

\myarticle[equipment]{Equipment}
All equipment shall be the property of the Oklahoma State University Department of Electrical and Computer Engineering. No equipment shall be removed from the club premises or from its location for station operation without the approval of the faculty advisor or two club officers. Any matters of access to the Ham Radio Office will be decided by the head of Electrical and Computer Engineering.

\myarticle[enforcement]{Enforcement}

\mysection[violations]{Violations}
If at any time the faculty advisor or a majority of the Executive Committee believes that a club has violated, attempted to violate, or intends to violate any law, university regulation, provision of this constitution, or club by-law in connection with the OSU ARC, that member may be put on immediate probation. The person or persons who put a member on probation may at any time terminate the probationary status of the member if the probation order was invoked in error.

\mysection[probation]{Probation}
A member on probation shall not have any of the privileges listed in \autoref{art:one} of this constitution, and at the option of the officers, may lose the right to vote. The length of the probationary period shall be determined by the officer team and faculty advisor.

\mysection[punitives]{Punitive measures}
If a member of the club violates any federal, state, or local law, they will be dealt with according to university and law enforcement authorities. Otherwise, they will be dealt with on a case by case basis by the officer team and faculty advisor. Penalties may include, but are not limited to the loss of full member privileges, the right to vote, access to club offices and equipment, and expulsion from the club.

\mysection[facultyauthority]{Authority of faculty advisor}
The faculty advisor has the right to remove anyone from the club at any time according to his or her discretion.

\myarticle[reports]{Reports to the University}

\mysection[reportofficers]{Officers}
Within two weeks of thee new officers entering office, the OSU ARC shall report a list of its officers and members to the Student Activities Office. The President, Treasurer and faculty advisor shall file a Signature Authorization Card in the same office.

\mysection[financialstatements]{Financial statements}
At the request of the Student Government Association, the OSU ARC shall submit a report of its finances and/or financial transactions.

\mysection[failuretocomply]{Failure to comply}
Failure to render such reports, either through continued negligence or willful omission may result in the suspension of the organization's charter.

\myarticle[subordination]{Subordination to the SGA Constitution}
Subordination to the Student Government Association Constitution This constitution shall not supersede any part or parts of the constitution of the Student Government Association of Oklahoma State University. This organization shall be under the jurisdiction of the Student Government Association.

\myarticle[interpconst]{Interpretation of this Constitution}
In the even that there is disagreement regarding the interpretation of this constitution, the officer team and faculty advisor shall attempt to resolve it first. If the officers and faculty advisor fail to reach a conclusion, the conflict shall be submitted to the Student Government Association and dealt with according to their constitution.

\myarticle[amendments]{Amendments}
Any new constitution, or any proposed amendment to this constitution, must be proposed in the exact written form to be considered in a regularly scheduled club meeting at least two weeks, but not more than six weeks, before it is to be voted upon. In voting on any new constitution or amendment to this constitution, a roll-call ballot of voting members shall be taken in open session. An acceptance vote of three-quarters of the total number of voting members must be given for passage. The new constitution or amendment shall not by valid until it is ratified by the Student Government.

\end{document}